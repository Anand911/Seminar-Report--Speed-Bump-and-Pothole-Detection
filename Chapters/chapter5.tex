\graphicspath{{Figures/chapter2}}
\chapter{CONCLUSION}
In this seminar, a real time embedded
system prototype has been proposed, which not only detects
the speed bump using vision camera but also utilizes the
best of its learning through CNN and applies the intelligence
to regulate its speed when such objects occurs. This proposed system can be implemented in
both autonomous and traditional vehicles. In an autonomous
car, it will be capable of controlling the speed of the vehicle
upon detection and for a traditional vehicle, it will warn the
driver about the upcoming speed bump or pothole. A dataset
was created based on the roads of Bangladesh and the data
was trained using the SSD Mobilenet algorithm. The model
was implemented on a Raspberry Pi with a single Raspberry
Pi camera. This system is capable of sending signals to other
devices like Arduino Uno which in turn can control the speed
of motors using a motor driver.Proposed
approach contains two module: first module deals with dataset
preparation, training and detection of speed bump; whereas
second module shows the operational performance over the
detected objects through its logical driving behavior. Test
results have revealed promising outcome and that too with
optimal power depletion


\noindent
 3D detection will be the future work of this system by
using a high-resolution infrared camera and more efficient
and enriched hardware including an enriched dataset and data
augmentation as well as capability
of the prototype is planned to be tested on various cognitive
techniques of optimization.Finally, the adopted strategy and
implementation is not restricted only to speed bump detection
task, but can also be adopted for other intelligent vehicle
applications where the detection of various challenging objects
are required.