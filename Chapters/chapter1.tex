\graphicspath{{Figures/chapter1}}
\chapter{INTRODUCTION}
\section{Genral Background}
specialized development in automobile sector has directed various organizations and institutions to device remote and autonomous vehicles such as intelligent vehicles, unmanned air vehicles (UAV) and drones, which are just the few examples of new generation in the transportation industry. Among all these examples where human is directly involved, safety is the primary concern especially in case of road transportation, as the chances of hazard cannot be ignored. According to the report of world health organization (WHO) estimated number of road traffic death are very high and causes of these situations are random but primarily includes casual driving and inability to understand the scene \cite{R3}. Besides, researchers have started to think on ideas of providing the safety in the form of better scene understanding capability through various sensors.

\noindent
Speed bumps and potholes are common road features that can pose challenges for autonomous vehicles.However, speed bumps and potholes can disrupt the smooth operation of these systems and potentially cause accidents if not properly detected and avoided.To address this issue, researchers and engineers are developing various techniques for detecting and avoiding speed bumps and potholes with autonomous vehicles. These approaches range from simple methods that use sensors to detect changes in road surface height, to more complex methods that employ machine learning algorithms to recognize and classify road features in real-time.
An IVS is expected to regulate its driving speed when it finds such object and ignore the chances of hazard. Understanding the features and detection of speed bump is little explored in existing research studies and had not been associated for the design of IVS \cite{R2}. In addition to the technical challenges, there are also legal and regulatory issues to consider when developing and deploying autonomous vehicles that can detect and avoid speed bumps and potholes. Governments and regulatory bodies around the world are working to establish guidelines and standards for the testing and operation of autonomous vehicles, including requirements for how they should handle road hazards such as speed bumps and potholes.

\noindent
The aim of this seminar is to present a feature descriptor for a speed bump detection model that is suitable for autonomous vehicles. The proposed model is based on a vision-based approach and has been designed to be low-cost and efficient. In addition, we have developed a real-time embedded system prototype to demonstrate the feasibility of implementing the model in autonomous cars. This research aims to contribute a practical solution for detecting speed bumps in real-time, which is a crucial capability for autonomous vehicles to navigate roads safely and efficiently.Our approach combines an artificial intelligence-based speed bump and pothole detection system with a microcontroller-based speed control system. We have chosen to use the Single Shot Multibox Detector (SSD) algorithm, as it is a lightweight algorithm that can run on a feed-forward convolutional network. This is in contrast to the Faster R-CNN algorithm, which uses a regional convolutional network and is not compatible with TensorFlow Lite, which is necessary for running on devices with limited computational power such as the Raspberry Pi \cite{R1}. Our solution aims to provide a practical and effective means for detecting and navigating road hazards in real-time with autonomous vehicles.
\section{Objective}
The goal of this project is to design and develop a system for detecting and safely navigating speed bumps and potholes on the road with autonomous vehicles using a multibox detector algorithm. This system will be designed to enhance the safety and efficiency of autonomous vehicle operation by accurately detecting road hazards and determining the appropriate speed and trajectory to navigate them safely. To achieve this goal, the project will involve implementing the multibox detector algorithm to analyze the detected speed bumps and potholes and determine the appropriate response. The algorithm will be integrated into an autonomous vehicle platform, and the performance of the system will be evaluated through testing on a variety of road surfaces and conditions.
\noindent
In addition to the technical challenges of developing an accurate and reliable system using the single shot multibox detector algorithm, the project will also focus on continually improving the robustness and reliability of the system through ongoing development and optimization. This will involve identifying and addressing potential issues or weaknesses in the system, and testing and refining the system to ensure it performs consistently in a range of real-world scenarios
\section{Relevance}
Accurate detection and safe navigation of speed bumps and potholes is critical for autonomous vehicle operation, as these road hazards can disrupt the smooth operation of the vehicle's sensors and navigation systems and potentially cause accidents if not properly detected and avoided. A system that can detect these obstacles in the road ahead and adjust the speed and trajectory of the vehicle accordingly could greatly improve the safety and efficiency of autonomous vehicle operation, particularly in regions with poor or varied road conditions where these types of obstacles are more likely to be encountered.

\noindent
The proposed project aims to make a significant contribution to the development of autonomous vehicle technology by developing a system for speed bump and pothole detection and speed control. This system could potentially have a major impact on the adoption and deployment of self-driving systems by providing a practical and effective solution for detecting and navigating road hazards in real-time. As such, this project is highly relevant to the seminar, as it addresses an important challenge in the field of autonomous vehicles and presents a potential solution that could have significant practical implications for the deployment and operation of self-driving systems