\begin{center}
    \LARGE\textbf{\underline{\textsf{APPENDIX-II}}}\\
    \vspace{5mm}
    \large\textbf{
  RAJAGIRI SCHOOL OF ENGINEERING AND TECHNOLOGY(AUTONOMOUS)\\\vspace{5mm}DEPARTMENT OF INFORMATION TECHNOLOGY\\PROGRAMME: INFORMATION TECHNOLOGY}

\end{center}
\par \noindent
\textbf{VISION}\\\noindent
To evolve into a department of excellence in information technology through the creation and exchange of knowledge through leading-edge-research, innovation, and services, which will, in turn, contribute towards solving complex societal problems and thus building peaceful and prosperous mankind.
\par \noindent
\textbf{MISSION}\\\noindent
To impart high-quality technical education, research training, professionalism, and strong ethical values to young minds for ensuring their productive careers in industry and academia so as to work with a commitment to the betterment of mankind.
\par \noindent
\textbf{PROGRAMOUTCOMES(PO)}\\\noindent
Information Technology program students will be able to:
\vspace{2mm}
\par \noindent
\textbf{PO 1. Engineering knowledge:} Apply the knowledge of mathematics, science, engineering fundamentals, and an engineering specialization to the solution of complex engineering problems.
\par \noindent
\textbf{PO 2. Problem analysis:} Identify, formulate, review research literature, and analyze complex engineering problems reaching substantiated conclusions using first principles of mathematics, natural sciences, and engineering sciences.
\par \noindent
\textbf{PO 3. Design/development of solutions:} Design solutions for complex engineering problems and design system components or processes that meet the specified needs with appropriate consideration for the public health and safety, and the cultural, societal, and environmental considerations.
\par \noindent
\textbf{PO 4. Conduct investigations of complex problems:} Use research-based knowledge and research methods including design of experiments, analysis and interpretation of data, and synthesis of the information to provide valid conclusions.
\par \noindent
\textbf{PO 5. Modern tool usage:} Create, select, and apply appropriate techniques, resources, and modern engineering and IT tools including prediction and modeling to complex engineering activities with an understanding of the limitations.
\par \noindent
\textbf{PO 6. The engineer and society:} Apply reasoning informed by the contextual knowledge to assess societal, health, safety, legal and cultural issues and the consequent responsibilities relevant to the professional engineering practice.
\par \noindent
\textbf{PO 7. Environment and sustainability:} Understand the impact of the professional engineering solutions in societal and environmental contexts, and demonstrate the knowledge of, and need for sustainable development.
\par \noindent
\textbf{PO 8. Ethics:} Apply ethical principles and commit to professional ethics and responsibilities and norms of the engineering practice.
\par \noindent
\textbf{PO 9. Individual and teamwork:} Function effectively as an individual, and as a member or leader in diverse teams, and in multidisciplinary settings.
\par \noindent
\textbf{PO 10. Communication:} Communicate effectively on complex engineering activities with the engineering community and with society at large, such as, being able to comprehend and write effective reports and design documentation, make effective presentations, and give and receive clear instructions.
\par \noindent
\textbf{PO 11. Project management and finance:} Demonstrate knowledge and understanding of the engineering and management principles and apply these to one’s own work, as a member and leader in a team, to manage projects and in multidisciplinary environments.
\par \noindent
\textbf{PO 12. Life-long learning:} Recognize the need for and have the preparation and ability to engage in independent and life-long learning in the broadest context of technological change.
\par\noindent
\textbf{PROGRAM SPECIFIC OUTCOMES (PSO)}\\\noindent
Information Technology program students will be able to:
\par\noindent
\textbf{PSO1:} Acquire skills to design, analyze and develop algorithms and implement them using high-level programming languages.
\par\noindent
\textbf{PSO2:} Contribute their engineering skills in computing and information engineering domains like network design and administration, database design and knowledge engineering.
\par\noindent
\textbf{PSO3:} Develop strong skills in systematic planning, developing, testing implementing and providing IT solutions for different domains which helps in the betterment of life.
\par\noindent
\textbf{PROGRAM EDUCATIONAL OBJECTIVES (PEO)}\\\noindent
Graduates of Information Technology program shall
\par\noindent
\textbf{PEO 1:} Have strong technical foundation for successful professional careers and to evolve as key-players / entrepreneurs in the field of information technology.
\par\noindent
\textbf{PEO 2:} Excel in analyzing, formulating and solving engineering problems to promote life-long learning, to develop applications, resulting in the betterment of the society.
\par\noindent
\textbf{PEO 3:} Have leadership skills and awareness on professional ethics and codes.
\par\noindent
%%%%%%%%%%tables
\renewcommand{\arraystretch}{2}
\textbf{COURSE OBJECTIVES:}
\vspace{3mm}
\begin{center}
\begin{tabular}{|m{2em}|m{35em}|}
\hline
1 & To do literature survey in a selected area of study.\\
\hline
2 & To understand an academic document from the literate and to give a presentation about it.\\
\hline
3 & To prepare a technical report.\\
\hline
\end{tabular}
\end{center}
\par \noindent
\textbf{COURSE OUTCOMES:}\\
After completion of the course the student will be able to
\vspace{3mm}

\begin{center}
    \begin{tabular}{|m{4em}|m{23em}|m{8em}|}
\hline
\textbf{CO.NO} & 
\textbf{DESCRIPTION} &
\textbf{Blooms' Taxonomy Level}\\
\hline
CO1 & Identify academic documents from the literature which are related to her/his areas of interest.
 & Level 3: Apply\\
 \hline
CO2 & Read and apprehend an academic document from the literature which is related to her/his areas of interest.& Level 4: Analyze\\
\hline
CO3 &Prepare a presentation about an academic document. & Level 6:\\
\hline 
CO4 & Give a presentation about an academic document. & Level 3: Apply\\
 \hline
CO5 & Prepare a technical report.& Level 6:Create\\
\hline
\end{tabular}
\end{center}
\vspace{5mm}
\textbf{CO-PO AND CO-PSO MAPPING}
\begin{center}
\footnotesize
    \begin{longtable}{|m{1.65em}|c|c|c|c|c|c|c|c|c|m{1.3em}|m{1.3em}|m{1.3em}|m{1.3em}|m{1.5em}|m{1.5em}|}
\hline
& PO 1 & PO 2 & PO3 & PO4 & PO5 & PO6 & PO7 & PO8 & PO9 & PO 10 & PO 11& PO 12 & PSO 1 & PSO 2 & PSO 3\\
\hline
CO1 & 2 & 2 & 1 & 1 & & 2 & 1 & & & & & 3 & 1 & &\\\hline
CO2 & 3 & 3 & 2 & 3 & & 2 & 1 & & & & & 3 & 1 & &\\\hline
CO3 & 3 & 2 & & & 3 & & & 1 & & 2 & & 3 & 2 & &\\\hline
CO4 & 3 & & & & 2 & & & 1 & & 3 & & 3 & & & 1\\\hline
CO5 & 3 & 3 & 3 & 3 & 2 & 2 & & 2 & & 3 & & 3 & & & 2\\
\hline
\end{longtable}
\end{center}