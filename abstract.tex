%\thispagestyle{empty}
\begin{center}
    \textsf{\Huge \textbf{ABSTRACT}
    }\\
    
\end{center}
\noindent
The development of self-driving cars has always been an extensive research field for the
automobile industry. Many problems must be overcome in order to create a capable self-driving car. Detection of road conditions is one of them. Potholes and speed bumps can cause accidents and have been a concern for drivers. In India, over 10,000 accidents have been reported due to these features. Image processing and acceleration data have been used to try and detect potholes. Detecting potholes and speed bumps can help us take a step toward reducing the number of accidents. Early detection of road irregularities can help to tune the active suspension of the vehicle, thus making the ride comfortable. Speed control of autonomous vehicles is also a major issue. A pothole is a downturn in a street surface; generally, blacktop asphalt, where traffic has eliminated broken bits of the asphalt, and speed humps are parabolic vertical traffic calming systems designed to slow traffic speeds. Hitting a pothole cannot just harm a vehicle's stuns and suspension; it can likewise make the driver lose control of their vehicle. Speed bumps are
useful for keeping vehicle speeds down and encouraging the driver to drive safely.
\par \noindent
The proposed method combines a microcontroller-based speed control system with a camera-based dash cam image retrieval system, as well as an artificial intelligence-based speed bump and pothole detection system to be employed in the construction of self-driving cars. A convolutional neural network model, a deep learning network design that learns directly from data, might be created to recognize potholes and speed bumps, eliminating the human feature extraction requirement. Deep convolutional neural networks (CNNs) have been investigated for many years to help detect potholes. Since the pothole and bumps are identified upfront, we can take preventive measures and make the ride safe and comfortable. The system is extended to send a signal to the speed controller unit of the car to reduce the speed on detection to avoid accidents or damage to the car. The developed car utilizes a single Raspberry Pi as its computational unit. In addition, the research investigates the behavior of economical hardware used to deploy deep learning models. The experimental results indicate that this model can achieve real-time response on a resource-constrained device without significant overheads, thus making it a viable option for autonomous driving in current intelligent transportation systems.